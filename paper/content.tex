% status: 100
% chapter: Amazon

\title{Amazon EMR}


\author{Scott Steinbruegge}
\affiliation{%
  \institution{Indiana University}
  \city{Bloomington} 
  \state{IN} 
  \postcode{47408}
  \country{USA}}
\email{srsteinb@iu.edu}


% The default list of authors is too long for headers}
\renewcommand{\shortauthors}{S. Steinbruegge}


\begin{abstract}
Amazon EMR is a Hadoop framework that allows users to process data on the AWS 
platform using their EC2 technology to spread the load across multiple EC2 
instances. Elasticity a major benefit of EMR as it can be set to auto scale 
up or down the number of EC2 instances that EMR is running in a cluster. The 
user can choose to run additional frameworks supported on EMR in addition to 
Hadoop, such as Spark, HBase, Flink and Presto. The platform allows the user 
to focus on the processing of the data and not have to deal with the setup, 
management or tuning of a Hadoop cluster. Using EMR allows a user to setup 
and provision a cluster quickly and allows for scalability of compute
resources up or down and in or out as needed. Interactions with EMR can
occur through a web service interface or by using the AWS Management Console
to launch and monitor clusters. To effectively configure and use EMR, 
knowledge of how EMR infrastructure is provisioned and how the EMR service
architecture works is beneficial.  
\end{abstract}

\keywords{hid521, hid-sp18-521, i524, Amazon, EMR}

\maketitle

\section{Introduction}

AWS has described EMR as ``a managed cluster platform that simplifies running 
big data frameworks, such as Apache Hadoop and Apache Spark, on AWS to process 
and analyze vast amounts of data. By using these frameworks and related 
open-source projects, such as Apache Hive and Apache Pig, you can process data 
for analytics purposes and business intelligence workloads
~\cite{hid-sp18-521-amazonemr-whatisemr}.'' EMR provides a variety of features
which make it an attractive platform to work with for many different big data
use cases. Elasticity, as with any cloud computing platform, is the major 
feature of EMR. Deploying a new cluster or resizing an existing cluster are 
both options. This elasticity allows the user to either automatically or 
manually scale an EMR cluster up or down based on processing needs. On demand 
cluster provisioning can also be done for scenarios when a user needs to run 
a one-time job, which can then be setup to shut down on completion in order 
to avoid paying unnecessary idle resource fees. But excess capacity never has 
to be provisioned based on a guess for future requirements and can be handled 
automatically based on CloudWatch metrics and a specified scaling 
policy~\cite{hid-sp18-521-amazonemr-details}.  

Multiple types of data stores can be integrated into EMR such as S3 (via EMR
File System), HDFS, DynamoDB, Redshift, Glacier and RDS. On premise data 
stores can also be used by EMR by integrating the AWS Data Pipeline, which 
can move data between on premise and AWS at chosen intervals. EMR fully 
supports Hadoop tools to which AWS has added modifications to that allow 
for improved interactions with other AWS services, some of which will be 
covered below in more detail the frameworks and applications 
section~\cite{hid-sp18-521-amazonemr-details}.  

Many other features are included that enhance usability. The type of EC2 
instances provisioned for an EMR cluster can be chosen based on the 
requirements of the use case, allowing the cluster to be built to handle 
standard, high CPU, high memory or high I/O based workloads. Additional 
configuration of EMR clusters can also be performed as root access can be 
granted to a user to perform further customization at the EC2 instance level. 
Amazon Cloudwatch can be used to monitor the performance of your EMR clusters 
and setup alarms on metrics you're interested in tracking. Deep learning 
frameworks are also supported on EMR within clusters that were created with 
a focus on increased GPU capabilities. Additional software can be installed 
on nodes within the cluster by using bootstrap scripts that execute when the 
cluster is launched. The EMR team at AWS also curates a Github repository of 
bootstrap scripts that can be used for installing and configuring additional 
software to EMR clusters~\cite{hid-sp18-521-amazonemr-details}. 

\section{Architecture}

A group of EC2 instances is what makes up an EMR cluster, which is the central 
component of EMR. Each EC2 instance in a cluster is called a node, of which 
there are three different node types. Every node type has different role and 
can have different software installed, allowing each node to play a specific 
role within the cluster that can then be utilized by any distributed 
applications the cluster is running. The master node is where management of 
the cluster occurs and contains the software that handles where data and tasks 
are dispersed between the other nodes, called slave nodes, for further 
processing. It also tracks cluster health and status of tasks that it has 
assigned. There are two types of slave nodes, core nodes and task nodes. Core 
nodes are where data in the HDFS is stored within the cluster and can contain 
additional software components which execute tasks. Task nodes are optional 
and are only allowed to run software to execute 
tasks~\cite{hid-sp18-521-amazonemr-overview}. 

Once a cluster is up and running, work can be submitted to it in a variety 
of ways: through the AWS Management Console, through the interface of the 
applications installed on the cluster, through APIs or using the AWS CLI. 
By using one of these methods, data can then be processed by either submitting 
jobs or queries directly to the specific applications installed on the cluster 
or by running ordered steps on the cluster. A step is considered a unit of 
work that tells the specified application how to manipulate the data. Data can 
be passed from one step to the next with each step calling a different 
application if needed. By default, when a step fails, all subsequent steps are 
then cancelled and no longer set to be run, but this behavior can be changed 
to ignore failures and continue with running the remaining 
steps~\cite{hid-sp18-521-amazonemr-overview}. 

Each EMR cluster follows a lifecycle during its creation. When a cluster is 
first provisioned with the applications the user has chosen, the cluster 
enters a starting state. Afterwards, the cluster goes into the bootstrapping 
state, which is where bootstrap actions such as additional software and any 
user-defined actions are then executed on the cluster. Once the bootstrapping 
state has completed, the cluster then enters the running state and is ready 
to accept and run jobs, queries or steps. If the cluster is set to 
automatically terminate after the completed steps, it will then enter the 
shutting down state which terminates all EC2 instances in the cluster and all 
data that was stored inside the cluster gets deleted. After a successful 
shutdown state, the cluster then moves to the completed state. Or if the 
cluster has auto terminate disabled, it then goes to the waiting state and 
stays online and ready to accept new commands at any time. This cluster would 
then need to be manually terminated whenever it is no longer needed. Any 
failures within the lifecycle of a cluster would cause the process to 
terminate the whole cluster including all of its EC2 instances and any data 
stored directly on the cluster. But there is a termination protection option 
that can be enabled which will allow the user to retrieve data from the 
cluster before it becomes fully 
terminated~\cite{hid-sp18-521-amazonemr-overview}. 

There are four layers of architecture which supply different functionality to 
an EMR cluster: storage, cluster resource management, data processing 
frameworks and applications and programs. The storage layer refers to the 
type of file system used by the cluster. Storage options include HDFS, EMR 
File System (EMRFS) and local file systems. ``Hadoop Distributed File System 
(HDFS) is a distributed, scalable file system for Hadoop. HDFS distributes the 
data it stores across instances in the cluster, storing multiple copies of 
data on different instances to ensure that no data is lost if an individual 
instance fails~\cite{hid-sp18-521-amazonemr-arch}.'' All HDFS storage is 
reclaimed upon cluster termination though. EMRFS adds to existing HDFS 
functionality by allowing direct access to data in S3, treating S3 as an 
extension of HDFS. It allows for the usage of persistent data instead of the 
ephemeral storage provided by HDFS. The local file system in an EMR cluster 
refers to disks connected to each of the EC2 nodes in the cluster. All EC2 
instances in a cluster come with pre-attached stored called an instance 
store, but this storage only exists as long as the instance is 
running~\cite{hid-sp18-521-amazonemr-arch}. 

The cluster resource management layer is where resources are managed and the
 scheduling of data processing jobs occurs. YARN, or Yet Another Resource 
Manager, is the default component used to handle these tasks in EMR because
 it can manage resources for a wide variety of the Apache Hadoop related data 
processing frameworks. The nodes in a cluster that administer YARN processes 
have an agent running of them that will track cluster health and talk directly 
with the EMR services~\cite{hid-sp18-521-amazonemr-arch}. 

\section{Frameworks and Applications}

The data processing framework handles the analysis and processing of the data. 
A variety of these frameworks utilize YARN and others supply their own 
resource management software. When choosing a framework, the use cases of the 
data need to be considered, such as batch, in-memory or streaming data 
processing. The framework chosen then determines what additional languages and 
interfaces can be used within the application layer. Three popular frameworks 
that are supported on EMR are MapReduce, Spark and 
Tez~\cite{hid-sp18-521-amazonemr-arch}. 

MapReduce is a programming model designed for distributed, parallel computing 
that helps simplify the creation of distributed applications by dealing with 
the logic, leaving the user to focus on the creation of Map and Reduce 
functions. Spark is an additional data processing framework used for 
distributed data processing, the main difference being that it performs 
in-memory caching of data sets whereas MapReduce uses 
disks~\cite{hid-sp18-521-amazonemr-arch}. Input, output and intermediate data 
are all stored in-memory allowing Spark to perform faster processing without 
I/O costs. Using Spark on EMR within EMRFS allows direct access to data stored 
on S3~\cite{hid-sp18-521-amazonemr-spark}. Tez is a framework that can be used 
as an alternative to MapReduce and uses directed acyclic graphs (DAG), which 
Spark also uses, for its data processing. DAGs allow applications that use the 
data processing framework to determine the overall workflow of a job before 
it is executed. It contains all of the steps within that workflow so than an 
optimal plan can be decided upon before execution. DAGs also allow for caching 
of intermediate processing results in-memory. MapReduce has to store these 
intermediate results on disk which can add to processing 
	overhead~\cite{hid-sp18-521-amazonemr-tez}. 

The final layer of EMR architecture is the applications layer. Applications 
residing on EMR can provide a variety of functionality depending on user 
needs such as allowing for the use of higher level programming languages 
with data processing, creation of streaming applications, machine learning 
and data warehouse implementation. Examples of applications that are supported 
include Flink, HBase, Hive, Mahout, MXNet, Pig, and Presto. Additional open 
source software is supported for applications that use their own cluster 
resource management layer instead of YARN. Other languages and libraries can 
be integrated into applications depending on the data processing framework. 
For example, MapReduce supports Java, Hive or Pig integration and Spark 
supports Spark SQL, Spark Steaming, MLib and 
GraphX~\cite{hid-sp18-521-amazonemr-arch}.  

\section{Conclusion}

Amazon EMR offers a cloud computing solution to handle a wide array of big 
data related use cases. A general understanding of the architecture and 
platforms supported on EMR allows for a greater ability to properly architect 
a solution to meet user demands. Provisioning and supporting all of hardware 
needed to implement a solution can all be done within the cloud, freeing up 
users to focus more time on working with the data itself. Elasticity of the 
platform is a major benefit of EMR, allowing the hardware to scale up or down 
automatically to meet the needs of the data processing. The primary component 
of EMR is a cluster, which is a collection of EC2 instances, also called 
nodes, with each node being assigned a specific task within the cluster. 
Each cluster contains four levels of architecture that provides the layers 
of EMR functionality: storage, cluster resource management, data processing 
frameworks and applications and programs. Multiple options can be chosen when 
implementing each of these layers, allowing the creation of different 
solutions for multiple use cases. Understanding the options available within 
each of these layers can help determine which applications will later be 
available for use with the EMR cluster. A high-level understanding of the 
architecture and implementation could then be applied to help dig deeper into 
the technologies and applications supported on EMR, allowing users to develop 
highly customized cloud based EMR platforms than can handle many different 
data processing challenges that companies and institutions are faced with 
today. 


\bibliographystyle{ACM-Reference-Format}
\bibliography{report} 

