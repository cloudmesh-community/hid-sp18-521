% status: 0
% chapter: TBD

\title{Performing Data Analysis on Amazon Web Services}

\author{Scott Steinbruegge}
\affiliation{%
  \institution{Indiana University}
  \city{Bloomington} 
  \state{IN} 
  \postcode{47408}
  \country{USA}}
\email{srsteinb@iu.edu}


% The default list of authors is too long for headers}
\renewcommand{\shortauthors}{S. Steinbruegge}


\begin{abstract}
Amazon Web Services provides a wide variety of database and data analysis
 services that allow a user to immediately start working on interpreting 
their data instead of trying to obtain or provision their own hardware 
beforehand. Using a data set that contains Medicare hospital comparison 
information, we will showcase how REST APIs can be used to directly interact 
with multiple AWS services that are used specifically for database or data 
analysis use cases. While there are multiple methods AWS supplies to interact 
with their services, we will be using the AWS Python SDK called 
Boto~\cite{hid-sp18-521-boto} in order to be able to work programmatically 
with the Amazon APIs exposed by each of their services so calls to them can 
then be made directly from our own REST APIs we develop. We will start with 
presenting how to get raw data onto the AWS platform and landing that data 
in S3 for use by other AWS services and then work our way up to newer, more 
advanced services such as SageMaker which is a machine learning service. 
Along the way we will provide explanations and hands on coding examples 
for each service we work with in order to gain insight into how to 
effectively interact with AWS while performing data analysis or other data 
related tasks. 
\end{abstract}

\keywords{hid-sp18-521, Amazon, AWS, REST API}


\maketitle


\bibliographystyle{ACM-Reference-Format}
\bibliography{report} 

