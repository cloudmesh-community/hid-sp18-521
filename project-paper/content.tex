% status: 0
% chapter: TBD

\title{Performing Data Analysis on Amazon Web Services}

\author{Scott Steinbruegge}
\affiliation{%
  \institution{Indiana University}
  \city{Bloomington} 
  \state{IN} 
  \postcode{47408}
  \country{USA}}
\email{srsteinb@iu.edu}


% The default list of authors is too long for headers}
\renewcommand{\shortauthors}{S. Steinbruegge}


\begin{abstract}
Amazon Web Services provides a wide variety of database and data analysis 
services that allow a user to immediately start working on interpreting their 
data instead of trying to obtain or provision their own hardware beforehand. 
Using a data set that contains Medicare hospital comparison information, we 
showcase how REST APIs can be used to directly interact with multiple AWS 
services that are used specifically for database or data analysis use cases. 
While there are multiple methods AWS supplies to interact with their services, 
the AWS Python SDK called Boto is used here in order to be able to work 
programmatically with the Amazon APIs exposed by each of their services so 
calls to them can then be made directly from our own REST APIs we develop. We 
start with presenting how to get raw data onto the AWS platform and landing 
that data in S3 for use by other AWS services and then work our way up to 
newer, more advanced services such as SageMaker which is a machine learning 
service. Along the way we provide explanations and hands on coding examples 
for each service in order to gain insight into how to effectively interact 
with AWS while performing data analysis or other data related tasks. 
\end{abstract}

\keywords{hid-sp18-521, Amazon, AWS, REST API}


\maketitle

\section{Interacting With AWS Services}

Between the AWS Management Console, SDKs (software development kits) and AWS 
CLI (command line interface), AWS provides multiple ways to interact with 
their services. While some of the work we perform is done through the 
Management Console, the majority of it is performed by the Python SDK for AWS 
named Boto. Using one of the AWS SDKs allows developers to be able to write 
their own software that interacts directly with AWS services through their own 
set of APIs. There are two levels of APIs that AWS provides, resource APIs and 
client APIs. The resource APIs provide an object-oriented interface to the 
services which expose multiple attributes and methods per service at a higher 
level than the lower level client API calls~\cite{hid-sp18-521-BotoResources}. 
Client APIs, also called low-level APIs, provide methods that map one-to-one 
with the service APIs. No level of abstraction is provided by the client 
APIs, the developer needs to explicitly provide all information about the 
targeted resource for every client API call~\cite{hid-sp18-521-BotoClients}. 
When available, the resource APIs are preferred since they abstract the 
majority of the backend details away from the developer and they can focus on 
interacting directly with the service using the necessary parameters. Boto 
also provides the ability to use waiters, which allow for polling the status 
of specific AWS resources within your code. An example of when a waiter could 
be used is when you run code trying to provision a new resource and moving 
onto the next step cannot be done until that new resource has been setup 
successfully. Using the waiter would allow your code to continuously poll the 
status of that dependent service~\cite{hid-sp18-521-pythonsdk}.


\bibliographystyle{ACM-Reference-Format}
\bibliography{report} 

