\section{AWS Greengrass}
AWS Greengrass~\cite{hid-sp18-521-Greengrass} lets the user run local compute, 
messaging, data caching, sync, and ML inference capabilities on connected
devices. Connected devices can then run AWS Lambda functions, keep device 
data in sync, and communicate with other devices, even when not connected
to the Internet. It is meant to extend AWS to devices so they can act 
locally on the data they generate, while still using the cloud for 
management, analytics, and storage. It can help you build IoT solutions 
that connect different types of devices with the cloud and each other. 
You can use a variety of languages and programming models to create your 
device software in the cloud, and then deploy it to your devices. It can 
also be programmed to filter device data and only transmit necessary 
information back to the cloud. Devices that run Linux and support ARM or 
x86 architectures can host the Greengrass Core which is what enables the
local execution of AWS Lambda code, messaging, data caching, and security.
The devices that are running AWS Greengrass Core can then act as a hub that 
can communicate with other devices that have the AWS IoT Device SDK 
installed. Those devices can then be configured to communicate with one 
another in a Greengrass Group. If the Greengrass Core device loses 
connectivity to the cloud, devices in the Greengrass Group can continue 
to communicate with each other over the local network. 
